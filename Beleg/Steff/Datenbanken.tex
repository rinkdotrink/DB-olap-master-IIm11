\section{Eingesetzte Datenbanken}
W�hrend der Arbeiten am Projekt musste entschieden werden welche  Datenbank benutzt werden sollte. Um dies zu entscheiden orientierten wir uns zun�chst an der Vorlesung, da wir in der darin genutzten Datenbank sicher sein konnten, dass s�mtliche ben�tigten Funktionen enthalten waren. Diese Datenbank war Oracle SQL. Eine andere Datenbank, auf die wir sp�ter erst ein Auge geworfen haben war MySQL, da sie neben der Oracle SQL ebenfalls alle n�tigen Funktionen enthielt.
\subsection{Oracle SQL}
Die erste Datenbank mit der wir gearbeitet haben war Oracle. Da sie bereits in der Vorlesung vorgestellt wurde. Es gibt sie in zwei Versionen:

\begin{itemize}
	\item Die kostenfreie Express-Version
	\begin{itemize}
		\item Unbegrenzt nutzbar
		\item Eingeschr�nkte Funktionen
		\item Leicht zu installieren
	\end{itemize}
	\item Die kostenpflichtige Enterprise-Version
	\begin{itemize}
		\item o	Nutzung ohne Lizenz auf 30 Tage begrenzt
		\item Voller Funktionsumfang
		\item Komplizierte Installation 
\end{itemize}


\end{itemize}
Beide Versionen konnten von uns genutzt werden, da man in der Express-Version bereits das Datenbankschema und die Abfragen zusammenstellen und testen konnte, bevor man sie in der Enterprise-Version mit Partitioning erg�nzt. So war es m�glich etwas Zeit zu sparen.
\subsection{MySQL}
Auf Grund von Installations- und Konfigurationsproblemen mit der Oracle-Datenbank haben wir sehr viel Zeit verloren. Daher wurde entschieden sich zus�tzlich MySQL anzusehen. Die Analyse der Funktionen ergab, dass wir ausschlie�lich die kostenfreie Version ben�tigten, was deshalb erstaunlich war, da MySQL von Oracle �bernommen worden ist.\newline
Neben dem Partitioning besitzt MySQL auch die M�glichkeit die Folgenden Indexe zu benutzen:

\begin{itemize}
	\item B-Tree
	\item R-Tree
	\item Hash-Table
\end{itemize}

Dar�ber hinaus wird die Freie MySQL-Datenbank mit einem Entwicklungstool ausgeliefert. Damit war es m�glich die Schemas in einem ERD zu designen und diese dann einfach mit einer beliebigen MySQL-Datenbank zu synchronisieren.

\subsection{Evaluierung und Entscheidung}
Somit kamen wir an den Punkt, an dem wir uns entscheiden mussten, welche der Datenbanken wir letztendlich f�r die Versuche benutzen. Abzuw�gen waren dabei diese Eigenschafften:

\begin{itemize}
	\item Funktionsumfang
	\item Kosten
	\item Installationskomplexit�t
\end{itemize}

Vom Funktionsumfang waren sich die kostenfreie Version von MySQL und die kostenpflichtige Version von Oracle ebenb�rtig. Bei den Kosten bzw. der Beschr�nkung die f�r die ben�tigten Funktionen vorhanden war MySQL dann im Vorteil. Die endg�ltige Entscheidung mit MySQL weiter zu arbeiten wurde jedoch getroffen, nachdem klar war das sich die MySQL-Datenbank schneller und einfacher installieren und konfigurieren lies.
