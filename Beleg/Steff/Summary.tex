\newpage
\section{Zusammenfassung} \label{Summary}
\begin{table}[h]
	\centering
		\begin{tabular}{|c|c|}
		\hline
		K4  & 0,993830822 \\ \hline
		K5  & 1,050671015 \\ \hline
		K6  & 0,911038608 \\ \hline
		K7  & 0,538090956 \\ \hline
		K8  & 1,13848872  \\ \hline
		K9  & 1,039717016 \\ \hline
		K10 & 0,975885368 \\ \hline			
		\end{tabular}
	\caption{Ver�nderung im Vergleich zu K3}
	\label{tab:VergleichZuK3}
\end{table}

In Tabelle \ref{tab:VergleichZuK3} sind die Faktoren abzulesen, um die sich die Abfragen im Durchschnitt verbessern oder verschlechtern. Da Konfiguration 3 (K3) bereits erheblich schneller ist als K1 oder K2 wurde sie dabei als Referenz benutzt. Bei der Analyse der Daten f�llt auf, dass f�r unsere Versuche die Konfiguration 8 mit 13\% Beschleunigung die Beste ist. Die schlechteste Konfiguration ist mit 47\% langsameren Abfragen Konfiguration 7. Dies gilt jedoch nur f�r das Szenario eines Onlineshops.

