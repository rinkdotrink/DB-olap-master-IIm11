\chapter{Ausblick} \label{Ausblick}
Da uns leider am Ende die Zeit fehlte um alle Versuche durch zu f�hren, die wir uns vorgenommen hatten, blieb einiges auf der Strecke. Auch gibt es noch viele andere M�glichkeiten die man nutzen k�nnte um seine Datenbanken zu optimieren. Auf einige dieser M�glichkeiten soll hier kurz eingegangen werden.\\

Zu den Dingen die wir leider nicht mehr testen konnten geh�rten einige der Datenbank-Engines. Da jede der Engines f�r verschiedene Probleme entwickelt wurde hat jede andere St�rken und Schw�chen. Gerade von der Memory-Engine erhofften wir uns noch bessere Ergebnisse. Dies weiter zu untersuchen w�re daher sicherlich lohnenswert.\\

Au�erdem ist es durchaus m�glich das Spektrum der Abfragen zu erweitern. Je mehr Abfragen man betrachtet um so besser sind die Aussagen die man am Ende �ber die Ergebnisse treffen kann. So fehlten uns beispielsweise noch Abfragen, in denen die Wirkung des Sub-Partitioning getestet werden konnte. Zum Beispiel h�tte man komplexe Abfragen einbauen k�nnen die Produkte auflisten, welche oft in gemeinsamen Warenk�rben auftauchen.\\

Eine andere Erweiterung dieser Arbeit k�nnte die Vergr��erung des Datenbankschemas durch weitere Tabellen sein. Dadurch w�ren zus�tzliche m�gliche Abfragen verf�gbar gewesen. Diese Handlung w�rde allerdings eine Erweiterung des Datengenerators nach sich ziehen.\\

Von allem bisher genannten M�glichkeiten besteht auch die Option mit weiteren Partitionierungsm�glichkeiten zu experimentieren um unter Umst�nden bessere Ergebnisse zu erzielen.