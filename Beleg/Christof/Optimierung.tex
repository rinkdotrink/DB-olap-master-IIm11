\section{Optimierungsma�nahmen}

\subsection{Indizes}
Bei einer Indizierung werden Objekte geordnet, was unter Umst�nden bei  Suchvorg�ngen enorme Geschwindigkeitsvorteile mit sich bringt. Ohne eine Indizierung m�sste die Datenbank die Suche bei dem ersten Datensatz beginnen und dann die gesamte Tabelle sequentiell nach passenden Datens�tzen durchforsten. Mit einem Index kann die Suche an einer bestimmten Position in der Mitte der Datendatei beginnen, was eine gezielte und schnelle Suche erm�glicht.

MySQL speichert die Indizes in B-Trees, unterst�tzt aber auch Hash-Indizes. Bei der Erstellung von Queries muss.. 

\subsubsection{B-Tree-Indizierung}
Ein B-Tree-Index eignet sich gut f�r Ausdr�cke mit Vergleichsoperatoren und beim dem BETWEEN-Operator. Auch ein LIKE-Vergleich ist m�glich wenn das Argument ein Konstanten-String ist, der nicht mit einem Jokerzeichen(=\%) beginnt.  Ein B-Tree-Index eignet sich auch f�r das Sortieren oder Gruppieren einer Tabelle.
MySQL kann auch mehrspaltige Indizes erstellen, dabei muss man jedoch beachten,  dass bei den Abfragen immer die erste Spalte des Indexes in einer WHERE-Klausel angegeben wird, sonst wird der Teil-Index nicht benutzt:
\begin{verbatim}
	z.B. INDEX name(nachname,vorname)
	
	Index wird benutzt 				-> SELECT *FROM test WHERE nachname="'Schmidt"'
	Index wird nicht benutzt  -> SELECT *FROM test WHERE vorname ="'Hans"'
\end{verbatim}
Das bedeutet, dass wenn beispielsweise ein dreispaltiger Index f�r (Spalte1,Spalte2,Spalte3) angelegt wird, wird er f�r (Spalte1),(Spalte1,Spalte2) und (Spalte1,Spalte2,Spalte3) eingesetzt. F�r (Spalte2) und f�r (Spalte2,Spalte3) kommt die Suchfunktionalit�t mit dem Index nicht zum Einsatz. Das muss man bei der Erstellung von SQL-Anweisungen beachten.
Au�erdem muss der Index alle AND-Ebenen in der WHERE-Klausel einbeziehen, sonst wird er ebenfalls nicht zur Optimierung benutzt:
\begin{verbatim}
	z.B. INDEX anschrift(nachname,vorname,ort)
	
	wird benutzt			  -> WHERE nachmame = "`Schmidt"' AND 
														vorname = "`Hans"' AND ort"'Halle"'	
	wird nicht benutzt  -> WHERE vorname ="'Hans"' AND ort ="'Buxtehude"'
\end{verbatim}
Wenn eine Abfrage auf einen gro�en Anteil von Datens�tzen zugreifen muss, kann es passieren, dass der Optimierer auf die Verwendung eines Indexes verzichtet und einen Full Table-Scan anordnet, da dadurch sich weniger Suchvorg�nge ergeben. Es sei denn, die Abfrage enth�lt eine LIMIT-Klausel und somit nur ein paar Datens�tze abgerufen werden sollen, wird trotzdem ein Index verwendet, da sich diese Datens�tze mit dem Index schneller finden lassen.
\subsubsection{Hash-Indizierung}
Ein Hash-Index unterscheidet sich grunds�tzlich dadurch, dass er nur vollst�ndige Schl�ssel zur Suche nach einem Datensatz verwenden kann. Au�erdem kann der Optimierer einen Hash-Index nicht zur Beschleunigung von ORDER-BY-Operationen verwenden und auch f�r Vergleiche mit  < > ist er nicht geeignet. Der Hash-Index eignet sich aber sehr gut f�r Vergleiche mit den Operatoren = oder < = >

\subsection{Partitionierung}
Bei der Partitionierung werden die Daten der einzelnen Tabellen nach bestimmten Regeln getrennt und auf mehrere physikalische Einheiten verteilt. Die Regel wird von dem Benutzer ausgew�hlt und als Partitionierungsfunktion bezeichnet. Es kann sich dabei um Modulus, um eine Hash-Funktion oder um einen Vergleich mit einer Menge von Wertebereichen oder Wertelisten handeln.
MySQL unterst�tzt nur die horizontale Partitionierung. Dabei werden die Tabellen zeilenweise getrennt. 

\subsubsection{Partitionierungsfunktionen in MySQL}
\begin{enumerate}
\item Range Partitioning \\
Den verschiedenen Partitionen werden Zeilen zugeordnet, deren Spaltenwerte einem bestimmten Wertebereich entsprechen. Beispielsweise die verkauften Produkte im ersten Quartal w�re eine Partition und die verkauften Produkte im zweiten Quartal w�re eine weitere Partition.

\item List Partitioning \\
Diese Methode ist dem Range Partitioning �hnlich. Der Unterschied ist nur der, dass die Kriterien nicht fortlaufend sein m�ssen und willk�rlich ausgew�hlt werden k�nnen. Es wird eine Entscheidungsliste f�r jede Partition erstellt. Beispielsweise die verkauften Produkte in den Jahren 02, 04, 06 w�re eine Partition und eine weitere Partition w�ren die verkauften Produkte in den Jahren 01, 03, 05.

\item Hash Partitioning und Linear Hash-Partitioning \\
Bei dieser Methode werden die Daten gleichm��ig verteilt. F�r den Hash gibt man haupts�chlich einen Spaltenwert an. Meistens wird dabei die Modulo-Funktion verwendet und bei dem linearen Hashing ein linearer Zweierpotenz-Algorithmus. 

\item Key Partitioning \\
Diese Partitionierung ist der  Hash-Partitionierung sehr �hnlich. Der Unterschied besteht darin, dass der Prim�rschl�ssel der Tabelle als Kriterium dient und dieser wird f�r die MD5-Operation verwendet.

\item Subpartitioning \\
Es besteht die M�glichkeit die partitionierten Tabellen noch weiter zu unterteilen.

\end{enumerate}
