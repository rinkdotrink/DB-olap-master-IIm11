\section*{Abstract}\label{Abstract}
Diese Arbeit besch�ftigt sich damit, wie sich die Performance von Abfragen steigern l�sst. Dabei wird nur der Bereich OLAP betrachtet. Der Datenimport mit Load sowie der Bereich OLTP werden nicht untersucht. Es wird nur auf Kaltstarts eingegangen, wie sie im Bereich Datawarehouse vorkommen. Performancesteigerungen mittels Caching bei Warmstarts sind nicht Gegenstand dieser Arbeit. Es wird untersucht, wie sich verschiedene Arten von Indexen und verschiedene Arten des Partitionings auf die Abfrageperformance von SQL-Queries auswirken. F�r die empirischen Messungen der Performance wird exemplarisch das Szenario eines Onlineshops gew�hlt, indem die Verk�ufe analysiert werden.\\
Diese Arbeit beantwortet die Frage, welche Datenbankkonfiguration f�r die gew�hl\-ten SQL-Abfragen im Durchschnitt die h�chste Abfrageperformance leistet. Ziel der Arbeit ist, Annahmen �ber die Performance verschiedener Konfigurationen zu treffen, diese theoretisch zu begr�nden und dann durch Messergebnisse praktisch zu belegen. Anhand von Messergebnissen werden die aufgestellten Hypothesen best�tigt oder wiederlegt. F�r wiederlegte Hypothesen wird eine Begr�ndung gesucht. Es wird davon ausgegangen, dass verschiedene Indexarten einen unterschiedlichen Einfluss auf bestimmte Queries haben werden. Es wird gezeigt, bei welchen Queries welche Indexarten bei welchen Spalten die Ausf�hrungszeit beschleunigen. Es wird angenommen, dass verschiedene Partitionierungsarten einen unterschiedlichen Einfluss auf bestimmte Queries haben werden. Es wird gezeigt, bei welchen Queries welche Partitionierungsarten auf welchen Spalten die Ausf�hrungszeit beschleunigen.\\
�ber die Ergebnisse der Arbeit l�sst sich zusammenfassend sagen, dass Hashindexe auf Prim�r- und Fremdschl�ssel und ein B-Tree auf das Datum die Beispiel-Abfragen im Durchschnitt um Faktor 910 beschleunigen, gegen�ber dem Weglassen s�mtlicher Indexe. Wird dar�ber hinaus Range-Partitioning verwendet, um Daten quartalsweise aufzuteilen, beschleunigt das die Beispiel-Abfragen im Durchschnitt nochmal um Faktor 1,14, gegen�ber der Verwendung von Hash-Indexen auf Prim�r- und Fremdschl�ssel. Wird dar�ber hinaus ein Subpartitioning - quartalsweise Aufteilung von Daten durch Range Partitioning und f�r jedes Quartal jeweils vier Hash-Partitions f�r weitere Daten -, beschleunigt das die gew�hlten Beispiel-Abfragen im Durchschnitt um den Faktor 1,04, gegen�ber der Verwendung von Hash-Indexen auf Prim�r- und Fremdschl�ssel.