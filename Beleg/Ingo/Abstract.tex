\section{Abstract}\label{Abstract}
In dieser Arbeit werden Datenbankkonfigurationen verschiedener Anwendungsf�lle f�r die ralationale Datenbank MySQL im Bereich OLAP (Online Analytical Processing) vorgestellt. Es wird gezeigt, wie sich Indexe, \index{Partitioning} und Warmstarts auf die Abarbeitungszeit von Queries auswirken. Datenbankanwendungen k�nnen sich, je nach Anwendungsfall, in ihren Anforderungen unterscheiden. Anforderungen sind z.B. kurze Antwortzeiten, hoher Datendurchsatz, Transaktionssicherheit oder geringer Speicherplatzverbrauch. In dieser Arbeit wird nur der Bereich OLAP behandelt. Der Bereich OLTP wird nicht untersucht. Diese Arbeit beantwortet die Frage, welche Datenbankkonfiguration zu welchem Anwendungsfall passt. Ziel der Arbeit ist, die Anwendungsf�lle auf geeignete Konfigurationen abzubilden, theoretisch zu begr�nden und durch Messergebnisse praktisch zu belegen. Es wird davon ausgegangen, dass verschiedene Indexarten einen unterschiedlichen Einfluss auf bestimmte Queries haben werden. Es wird gezeigt, bei welchen Queries welche Indexarten bei welchen Spalten die Ausf�hrungszeit beschleunigen. Ebenso wird angenommen, dass ein Warmstart gegen�ber einem Kaltstart Queries wesentlich beschleunigen wird. Es wird angenommen, dass verschiedene Partitionierungsarten einen unterschiedlichen Einfluss auf bestimmte Queries haben werden. Es wird gezeigt, bei welchen Queries welche Partitionierungsarten auf welchen Spalten die Ausf�hrungszeit beschleunigen. [todo: Ergebnisse...]