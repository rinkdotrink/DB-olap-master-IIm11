\section{EasyMock}\label{EasyMock}
Um den Datengenerator zu testen, wird EasyMock 3.0 verwendet. EasyMock ist ein Test-Framework zur dynamischen Generierung von Mock Objekten f�r Schnittstellen und Klassen. Mock-Objekte werden f�r Unit-Tests von Java-Programmen verwendet.

\subsection{Was ist ein Mock-Objekt?}
Beim Unit-Testing kollaborieren Units mit anderen Units. Die Kollaborateure werden durch Mock Objekte simuliert. Im Gegensatz zu einem Stub �berpr�ft das Mock-Objekt, ob es wie erwartet verwendet wurde. In einem Unit-Test werden Klassen bzw. Methoden isoliert von ihrer Umgebung getestet. Um die Testobjekte isoliert zu testen, m�ssen die Schnittstellen des Testobjekts durch Mock-Objekte ersetzt werden. Die Mock-Objekte sind Platzhalter f�r die echten Objekte. Das Verhalten eines dynamischen Mock-Objekts wird nicht in einer Klasse programmiert, sondern von dem Unit-Test aufgezeichnet. Es m�ssen keine Klassen von Hand geschrieben werden. Es muss auch kein Quellcode der Mock-Klassen, mit den echten Klassen synchron gehalten werden. Mock-Objekte werden bei EasyMock ''on the fly'' generiert, sind sicherer gegen Refactoring und damit besonders f�r Test Driven Development geeignet.

\subsection{Wie wird EasyMock benutzt?}
Um EasyMock zu benutzen, sind folgende Schritte n�tig:
\begin{enumerate}
	\item das Mock-Objekt von der Klasse / Schnittstelle die simuliert werden soll, erzeugen und dem zu testenden Objekt �bergeben,
	\item das erwartete Verhalten aufzeichnen,
	\item das Mock-Objekt auf Wiedergabe-Modus stellen,
	\item verifizieren, ob das Mock-Objekt auch so benutzt wurde, wie in Schritt zwei spezifiziert
\end{enumerate}
\cite{bib3}