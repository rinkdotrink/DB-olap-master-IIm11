\section{Relevanz des Forschungsgegenstandes}
Der Forschungsgegenstand in dieser Arbeit ist es, geeignete Konfigurationen f�r verschiedene, konkrete,
praxisrelevante Anwendungsf�lle zu finden. Der Forschungsgegenstand ist relevant, da bisher keine konkreten Werte, die die Performance der gew�hlten Konfigurationen beschreibt, vorliegen. Ziel dieser Foschung ist es, die Annahmen f�r Konfigurationen zu treffen, die Performance der Konfigurationen zu messen und dann zu interpretieren, ob die Annahmen sich mit den Messergebnissen best�tigen oder wiederlegt werden. Die Herausforderung dieser Arbeit ist, geeignete Anwendungsf�lle aus verschiedenen Anforderungen von Anwendungen zu finden. F�r jede dieser Anwendungsf�lle die geeignetste Kongiguration zu w�hlen. F�r diese Konfiguration werden dann Annahmen getroffen und Performancewerte gemessen. Die Annahmen werden dann mit Hilfe der Messergebnisse verifiziert bzw. falsifiziert. Um die Performance von Konfigurationen messen zu k�nnen, muss ein Testdatengenerator angefertigt werden. Dabei m�ssen technische Probleme gel�st werden. Um die optimale Konfiguration f�r einen Anwendungsfall zu finden, muss sich vertiefend in eine Datenbanken eingearbeitet werden. Das geschieht z.B. unter Zuhilfenahme von B�chern und Online-Ressourcen. In diesen Medien ist der Forschungsstand zur Erstellung von Konfigurationen dokumentiert.