\section{Relevanz des Forschungsgegenstandes}\label{RelevanzDesForschungsgegenstandes}
Der Forschungsgegenstand dieser Arbeit ist, Annahmen �ber die Abfrageperformance von Queries auf verschiedenen Konfigurationen zu treffen, die Performance zu messen und gegebenfalls wiederlegte Annahmen zu erkl�ren.\\
Der Forschungsgegenstand ist relevant, da bisher keine konkreten Werte vorliegen, die die Performance der gew�hlten Konfigurationen in Abh�ngigkeit der Abfragen beschreibt. Ziel dieser Foschung ist es, eine Konfiguration zu finden, die im Durchschnitt die beste Performance f�r alle Abfragen bringt. Um die optimale Konfiguration f�r alle Anwendungsf�lle zu finden, muss sich vertiefend in eine Datenbanken eingearbeitet werden. Das geschieht z.B. unter Zuhilfenahme von B�chern und Online-Ressourcen. In diesen Medien ist der Forschungsstand zur Erstellung von Konfigurationen dokumentiert.
Bei der Anfertigung des Datengenerators m�ssen dar�berhinaus technische Probleme gel�st werden.