\section{Einleitung}\label{Einleitung}

%\newglossaryentry{EasyMock}{name=EasyMock,description={Test-Framework zur dynamischen Generierung von Mock Objekten}}
%\newglossaryentry{Dependency Injection}{name=Dependency Injection,description={Entwurfsmuster, um Abh�ngigkeiten zwischen Objekten zu %minimieren}}

%\newacronym{IDE}{IDE}{Integrated Development Environment}

Ziel dieses Projektes ist die Performance eines relationalen Datenbankmanagementsystems f�r bestimmte Datenbank-Konfigurationen zu messen und auszuwerten. Eine Datenbank-Konfiguration, im Folgenden kurz Konfiguration genannt, ergibt sich aus den Anforderungen, die eine Anwendung an das DBMS bzw. die konkrete Datenbank stellt. 
Es ergibt sich die Frage, f�r welche konkrete Anforderung welche konkrete Konfiguration gew�hlt werden sollte? Um diese Frage zu beantworten, sind mehrere Schritte n�tig. Es werden Anforderungen aus vier verschiedenen praktischen Anwendungsf�llen gesammelt. Es wird dann eines der bekanntesten relationalen DBMS exemplarisch ausgew�hlt. Welches es genau sein wird, ist f�r die Aufgabe zweitrangig. Dann wird ein Anwendungsfall aus den vier Anwendungsf�llen ausgew�hlt. F�r ihn wird im DBMS ein ERD angefertigt, dass die Tabellen beschreibt. Um die Performance f�r die vier Anwendungsf�lle zu messen, werden die Tabellen aus dem ERD in unterschiedlich konfigurierten Datenbanken getestet. Aus Gr�nden der Einfachheit und �bersichtlichkeit wird dieses eine ERD auch f�r die Konfigurationen der drei anderen Andwendungsf�lle genutzt. Weil es in diesem Projekt um die Performance verschiedener Konfigurationen geht, muss das ERD nicht zu jedem Anwendungsfall passen. Um die Performance zu messen, m�ssen die Tabellen vorher mit Testdaten bef�llt werden. Dazu ist ein Datengenerator zu erstellen. Er generiert die Testdaten f�r die Tabellen. Die konkreten Anforderungen, die an den Datengenerator gestellt werden, m�ssen analysiert werden. Dann wird der Generator entworfen, implementiert und getestet. F�r den Datengenerator wird ein Build Mangagement Tool eingesetzt sowie Frameworks f�r Dependency Injection und zum Testen der Anwendung. Die Zeiten f�r das Bef�llen der Tabellen mit dem Generator werden gemessen und ausgewertet. Ebenso werden f�r jede der vier Konfigurationen verschiedene Queries auf den mit Testdaten bef�llten Tabellen gefahren und die Antwortzeiten gemessen. Die Messungen erfolgen alle auf einem Referenzsystem. So k�nnen verschiedene Konfigurationen �ber ihre Performancewerte miteinander verglichen werden.