\subsection{Einleitung}\label{Einleitung}
Ziel dieses Projektes ist die Performance eines relationalen Datenbankmanagementsystems f�r bestimmte Konfigurationen zu testen. Die Konfiguration ergibt sich aus den Anforderungen, die an das DBMS bzw. die konkrete Datenbank gestellt werden. 
Es ergibt sich die Frage, f�r welche konkrete Anforderung welche konkrete Konfiguration gew�hlt werden sollte? Um diese Frage zu beantworten, sind mehrere Schritte n�tig. Es wird zuerst ein DBMS exemplarisch ausgew�hlt. Dann wird f�r dieses DBMS ein ERD angefertigt, dass die Tabellen beschreibt. Um die Performance zu testen, muss ein Datengenerator geschrieben werden, der die Testdaten f�r die Tabellen generiert. Die konkreten Anforderungen, die an den Datengenerator gestellt werden, m�ssen analysiert werden. Dann wird der Generator entworfen, implementiert und getestet. F�r den Datengenerator wird ein Build Mangagement Tool eingesetzt sowie Frameworks f�r Dependency Injection und zum Testen der Anwendung. Dann werden f�r die Konfigurationen Queries auf den mit Testdaten bef�llten Tabellen gefahren und die Antwortzeiten gemessen. Die Messungen erfolgen alle auf einem Referenzsystem. So k�nnen verschiedene Konfigurationen �ber ihre Performancewerte miteinander verglichen werden.