\section{Dependency Injection}\label{DependencyInjection}
Der Datengenerator verwendet das Architektur-Muster Dependency Injection um Abh�ngigkeiten zwischen Objekten zu minimieren. Dabei k�mmert sich das Objekt nicht mehr selbst um die Erzeugung seiner abh�ngigen Objekte. Diese Abh�ngig\-keiten werden von einem Framework erstellt. Der Code des Objektes wird unabh�ngig von seiner Umgebung. Dadurch wird das Objekt leichter unit-testbar, da Abh�ngigkeiten zentral verf�gbar sind.

\subsection{Dependency Injection mit Google Guice}
In dem Datengenerator wird das Dependency Injection Framework Google Guice verwendet. Eine Abh�ngigkeit wird am besten �ber einen Konstruktor in ein Objekt injiziert. Daf�r braucht der Konstruktor nur mit @Inject annotiert werden.

\lstset{language=C,caption={Dependency Injection �ber einen Konstruktor},label=GoogleGuice1}
\lstinputlisting[language=Java]{Ingo/Listings/GoogleGuice.java}

In der configure()-Methode wird Google Guice u.a. bekannt gemacht, welche Implementierung von IDBWriter und Creator injiziert werden sollen.

\lstset{language=Java,caption={configure},label=configure}
\lstinputlisting[language=Java]{Ingo/Listings/DBModule.java}

Da die Main()-Methodedas nicht von Guice instanziiert wird, kann in ihm keine Constructor Context Injection verwendet werden. Um in Main.java selbst Klassen erzeugen zu k�nnen, die mit Guice verwaltet werden, wird com.google.inject.Injector benutzt.

\lstset{language=Java,caption={Injector},label=Injector}
\lstinputlisting[language=Java]{Ingo/Listings/Main.java}

\cite{bib5}