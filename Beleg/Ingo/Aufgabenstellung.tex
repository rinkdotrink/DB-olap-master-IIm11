\section{Aufgabenstellung}\label{Aufgabenstellung}
In diesem Projekt soll die Abfrage-Performace f�r ein bestimmtes Szenario gemessen werden. Dazu ist ein Szenario auszuw�hlen, das im Bereich OLAP und Data Warehouse angesiedelt ist. F�r das Szenario sind geeignete Konfigurationen zu w�hlen. Konfigurationen unterscheiden sich in Art- und Anzahl von Indexen und in den Partitionierungsarten. Es wird angenommen, dass jede Konfiguration f�r eine bestimmte Art von Abfragen besonders geeignet ist. Es sollen verschiedenartige Abfragen gew�hlt werden, f�r die jeweils die Performance gemessen wird. Um die Performance messen zu k�nnen, ist ein ERD f�r ein gew�hltes Szenario anzufertigen. Die Tabellen des ERD sollen mit Testdaten gef�llt werden. Dazu ist ein Datengenerator anzufertigen. Die Abfragen werden auf die gef�llten Tabellen angewendet. Die Zeit, die ein Query braucht, um auf einer bestimmten Konfiguration ausgef�hrt zu werden, wird gemessen. Das Messergebnis wird mit der Annahme verglichen und ausgewertet.