\subsection{Implementierung und Test}
Das Projekt liegt als Maven-Eclipse-Projekt unter git@github.com:rinkdotrink/DB-olap-master-IIm11.git

\subsection{Clean Code}
Das Projekt folgt dem Clean Code Ansatz, dass unter anderem besagt, dass Variablen und Methoden nicht kommentiert werden, sondern sich selbst beschreiben sollen. Bei diesem Ansatz wird dringend empfohlen, das jede Methode immer nur eine einzige Aufgabe erledigt.

Die Anwendung wurde Test Driven implementiert. Zu jeder �ffentlichen Methode wurde vor ihrer Implementierung ein Test geschrieben.

Es wurde all das, was ver�nderlich ist, wie z.B. die Generatoren f�r Kunden, Produkte etc. von den gleich bleibenden Teilen in der Generator-Klasse getrennt. Damit sind die Klassen f�r Erweiterungen offen, aber f�r Ver�nderungen geschlossen. D.h. Klassen k�nnen erweitert werden um neues Verhalten hinzuzuf�gen, aber ohne bestehenden Code zu ver�ndern (open/closed principle).

Es wurde gegen Schnittstellen und nicht auf Implementierungen programmiert. Dadurch sind die Objekte lose gekoppelt und der Code ist offen f�r Ver�nderungen, da die Objekte untereinander nur Schnittstellen kennen und sonst wenig Kenntnisse voneinander haben.

Es wurde darauf geachtet, dass jede Klasse nur eine einzige Verantwortlichkeit hat (hohe Koh�sion).

Es wurde nur der Code vererbt, der sich mit hoher Wahrscheinlichkeit nicht �ndert.

\subsubsection{Projektumgebung}
Eclipse-IDE 3.7, JDK 1.7, JUnit 4.8.2, Maven 3, Guice 3, EasyMock 3.0, Log4J 1.2.16, mysql-connector-java 5.1.6, Umlet 11.3

\subsubsection{Projekt aus dem repository laden}
Berechtigungen
git pull

\subsubsection{Maven}

\subsubsection{Maven-Projekt in Eclipse importieren}



\subsubsection{Maven Projekt ausf�hren}
Entweder �ber Jar oder �ber Eclipse

\subsubsection{Jar mit allen Abh�ngigkeiten erstellen}
package


\subsection{Implementierung der Funktionalit�t}
Die drei Komponenten

\subsection{Unit Test mit EasyMock 3.0}