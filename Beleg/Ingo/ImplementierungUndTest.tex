\section{Implementierung und Test}
Das Projekt liegt als Maven-Eclipse-Projekt unter: \\
git@github.com:rinkdotrink/DB-olap-master-IIm11.git

\subsection{Clean Code}
Das Projekt folgt dem Clean Code Ansatz, welches unter anderem besagt, dass Variablen und Methoden nicht kommentiert werden, sondern sich selbst beschreiben sollen. Bei diesem Ansatz wird dringend empfohlen, dass jede Methode immer nur eine einzige Aufgabe erledigt.

Die Anwendung wurde Test Driven implementiert. Zu jeder �ffentlichen Methode wurde vor ihrer Implementierung ein Test geschrieben.

Es wurde all das, was ver�nderlich ist, wie z.B. die Generatoren f�r Kunden, Produkte etc. von den gleich bleibenden Teilen in der Generator-Klasse getrennt. Damit sind die Klassen f�r Erweiterungen offen, aber f�r Ver�nderungen geschlossen. D.h. Klassen k�nnen erweitert werden um neues Verhalten hinzuzuf�gen, aber ohne bestehenden Code zu ver�ndern (open/closed principle).

Es wurde gegen Schnittstellen und nicht auf Implementierungen programmiert. Dadurch sind die Objekte lose gekoppelt und der Code ist offen f�r Ver�nderungen, da die Objekte untereinander nur Schnittstellen kennen und sonst wenig Kenntnisse voneinander haben.

Es wurde darauf geachtet, dass jede Klasse nur eine einzige Verantwortlichkeit hat (hohe Koh�sion).

Es wurde nur der Code vererbt, der sich mit hoher Wahrscheinlichkeit nicht �ndert.
\cite{bib1}

\subsection{Projektumgebung}
Eclipse-IDE 3.7, JDK 1.7, JUnit 4.8.2, Maven 3, Guice 3, EasyMock 3.0, Log4J 1.2.16, mysql-connector-java 5.1.6, Umlet 11.3

\subsection{Maven}
Um Java-Programme standardisiert erstellen und verwalten zu k�nnen, wird Apache Maven eingesetzt.

\subsection{Maven-Projekt in Eclipse importieren}
Das Projekt muss von github gepullt werden und kann dann in Eclipse unter \texttt{File -> Import -> Maven -> Existing Maven Projects} importieren werden.

\subsection{Maven Projekt ausf�hren}
In Eclipse erst auf das Projekt klicken und dann auf \texttt{Run As -> Run Configurations..} klicken. Links in den Konfigurationen auf Java Application doppelklicken. Unter Arguments m�ssen die vier �bergabeparameter eingetragen werden.

\subsection{Bugtracker}
Der Bugtracker ist zu finden unter:\\
\texttt{https://github.com/rinkdotrink/DB-olap-master-IIm11}