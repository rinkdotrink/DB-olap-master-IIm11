\section{Szenarien f�r Datenbankanwendungen}\label{AnwendungsfaelleFuerDatenbankanwendungen}

\begin{itemize}
\item \textbf{Szenario 1}\\
Ein Online-Shop will seine Bestellungen �ber eine MySQL-Datenbank abwickeln. Das Management m�chte auf der selben Datenbank Abfragen auf bestimmten Tabellen t�tigen, wie z.B.:

\begin{itemize}
\item Welches Produkt wurde wie oft gekauft?
\item Wieviel Umsatz wurde von wem in einem bestimmtem Zeitraum generiert?
\item Wie viele Kunden haben in einem bestimmten Zeitraum bestellt?
\end{itemize}

In den Tabellen befinden sich mehrere Millionen Tupel und der Arbeitsspeicher die nicht in den Arbeitsspeicher des Servers passen. die Abfragen k�nnen auch �ber Nacht auf dem Produktivsystem laufen. (innoDB)

\item \textbf{Szenario 3}\\
Ein bekannter Blogger m�chte das Weblog-System WordPress einsetzen um Beitr�ge zu ver�ffentlichen. Die Kommentare und Blog-Posts werden strukturiert gespeichert. Zu bestimmten Spitzenzeiten laden hunderte Leser gleichzeitig seine Blogposts. (MyISAM)

\item \textbf{Szenario 4}\\
In einer Erdbebenwarnstation sollen Bodenersch�tterungen durch einen Seismografen erfasst und digital archiviert werden. Der Seismograf misst 50 mal in der Sekunde. Es sollen auch feinste Amplitudenausschl�ge erfasst werden. (Archive)
\end{itemize}