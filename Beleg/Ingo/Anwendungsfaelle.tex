\section{Szenarien f�r Datenbankanwendungen}\label{AnwendungsfaelleFuerDatenbankanwendungen}

Im Folgenden werden verschiedene Szenarien f�r Datenbankanwendungen gegeben. F�r jedes Szenario bietet MySQL eine speziell optimierte Engine. Engines verwalten z.B. Transaktionen, Indexe und referenzielle Integrit�t. Im Kapitel \ref{Speichermaschinen} auf Seite \pageref{Speichermaschinen} wird darauf eingegangen, welche Eigenschaften Engines besitzen.

\begin{itemize}
\item \textbf{Szenario 1 -  Engine: InnoDB}\\
Ein Online-Shop will seine Bestellungen �ber eine MySQL-Datenbank abwickeln. Das Management m�chte auf der selben Datenbank Abfragen auf bestimmten Tabellen t�tigen, wie z.B.:

\begin{itemize}
\item Welches Produkt wurde wie oft gekauft?
\item Wieviel Umsatz wurde von wem in einem bestimmtem Zeitraum generiert?
\item Wie viele Kunden haben an einem bestimmten Zeitpunkt bestellt?
\item Wer hat wieviel an den gegebenen neun Tagen umgesetzt?
\end{itemize}

In den Tabellen befinden sich mehrere Millionen Tupel. Diese passen nicht in den Arbeitsspeicher des Servers der Firma. Die Abfragen sollen �ber Nacht auf dem Produktivsystem laufen.

\item \textbf{Szenario 2 - Engine: Memory}\\
Eine Softwareunternehmen das Browserspiele entwickelt, m�chte von den Spielerdaten Statistiken erstellen. Ihnen steht eine Serverinfrastruktur f�r In-Memory-Computing zur Verf�gung.

\item \textbf{Szenario 3 - Engine: MyISAM}\\
Ein bekannter Blogger m�chte das Weblog-System WordPress einsetzen um Beitr�ge zu ver�ffentlichen. Die Kommentare und Blog-Posts werden strukturiert gespeichert. Zu bestimmten Spitzenzeiten laden hunderte Leser gleichzeitig seine Blogposts.

\item \textbf{Szenario 4 - Engine: Archive}\\
In einer Erdbebenwarnstation sollen Bodenersch�tterungen durch einen Seismografen erfasst und digital archiviert werden. Der Seismograf misst 50 mal in der Sekunde. Es sollen auch feinste Amplitudenausschl�ge erfasst werden. 
\end{itemize}

Um die Performance von Konfigurationen zu messen, wird das Szenario 1 und damit die Engine InnoDB gew�hlt. Es k�nnte nat�rlich auch jedes andere Szenario gew�hlt und weiter untersucht werden.