\documentclass[a4paper,11pt]{article}

\usepackage{array}
\usepackage{graphicx}
\usepackage[latin9]{inputenc}
\usepackage[german]{babel}
%\usepackage{ngerman}
\usepackage[T1]{fontenc}
\usepackage[left=3cm,right=2cm]{geometry}
\usepackage[colorlinks=false, pdfborder={0 0 0}]{hyperref}	%Links im Inhaltsverzeichniss
\usepackage{pdflscape}
%\usepackage{lscape}
\usepackage{here} % hiermit k�nnen die Bilder dort platziert werden, wo man sie haben m�chte!!! Verwendung: \begin{figure}[H]...\end{figure}



%\lstset{numbers=left, numberstyle=\tiny, numbersep=5pt}
%\lstset{language=Java}
%\usepackage{makeidx}
%\makeindex

%\usepackage[acronym,toc, nonumberlist]{glossaries}
%\newglossary[slg]{symbols}{sym}{sbl}{List of Symbols}

%\addto\captionsngerman{%
% \renewcommand\glossaryname{Glossar}}
%\deftranslation[to=ngerman]{Acronyms}{Abk�rzungsverzeichnis}
%\deftranslation[to=ngerman]{Glossary}{Glossar}


%\glossarystyle{listdotted}

%\makeglossaries

\begin{document}
\newpage

\title{Thema der Arbeit}
\author{
Christof Ochmann\\ Matrikelnummer: 35989 \\Hochschule Zittau/G�rlitz\\ Fakult�t Elektrotechnik und Informatik\\ IIm11
\and
Stefan L�ttke \\ Matrikelnummer: xxxxx \\Hochschule Zittau/G�rlitz\\ Fakult�t Elektrotechnik und Informatik\\ IIm11
\and
Ingo K�rner \\ Matrikelnummer: 40586 \\Hochschule Zittau/G�rlitz\\ Fakult�t Elektrotechnik und Informatik\\ IIm11
}
\date{2012-02-02}

\maketitle
\newpage

\tableofcontents
\newpage

\part{Wohnheimprojekt}

\part{Performance-Messungen von Datenbank Konfigurationen}

\section{Theorie}
\section{Einleitung}\label{Einleitung}

%\newglossaryentry{EasyMock}{name=EasyMock,description={Test-Framework zur dynamischen Generierung von Mock Objekten}}
%\newglossaryentry{Dependency Injection}{name=Dependency Injection,description={Entwurfsmuster, um Abh�ngigkeiten zwischen Objekten zu %minimieren}}

%\newacronym{IDE}{IDE}{Integrated Development Environment}

Ziel dieses Projektes ist die Performance eines relationalen Datenbankmanagementsystems f�r bestimmte Datenbank-Konfigurationen zu messen und auszuwerten. Eine Datenbank-Konfiguration, im Folgenden kurz Konfiguration genannt, ergibt sich aus den Anforderungen, die eine Anwendung an das DBMS bzw. die konkrete Datenbank stellt. 
Es ergibt sich die Frage, f�r welche konkrete Anforderung welche konkrete Konfiguration gew�hlt werden sollte? Um diese Frage zu beantworten, sind mehrere Schritte n�tig. Es werden Anforderungen aus vier verschiedenen praktischen Anwendungsf�llen gesammelt. Es wird dann eines der bekanntesten relationalen DBMS exemplarisch ausgew�hlt. Welches es genau sein wird, ist f�r die Aufgabe zweitrangig. Dann wird ein Anwendungsfall aus den vier Anwendungsf�llen ausgew�hlt. F�r ihn wird im DBMS ein ERD angefertigt, dass die Tabellen beschreibt. Um die Performance f�r die vier Anwendungsf�lle zu messen, werden die Tabellen aus dem ERD in unterschiedlich konfigurierten Datenbanken getestet. Aus Gr�nden der Einfachheit und �bersichtlichkeit wird dieses eine ERD auch f�r die Konfigurationen der drei anderen Andwendungsf�lle genutzt. Weil es in diesem Projekt um die Performance verschiedener Konfigurationen geht, muss das ERD nicht zu jedem Anwendungsfall passen. Um die Performance zu messen, m�ssen die Tabellen vorher mit Testdaten bef�llt werden. Dazu ist ein Datengenerator zu erstellen. Er generiert die Testdaten f�r die Tabellen. Die konkreten Anforderungen, die an den Datengenerator gestellt werden, m�ssen analysiert werden. Dann wird der Generator entworfen, implementiert und getestet. F�r den Datengenerator wird ein Build Mangagement Tool eingesetzt sowie Frameworks f�r Dependency Injection und zum Testen der Anwendung. Die Zeiten f�r das Bef�llen der Tabellen mit dem Generator werden gemessen und ausgewertet. Ebenso werden f�r jede der vier Konfigurationen verschiedene Queries auf den mit Testdaten bef�llten Tabellen gefahren und die Antwortzeiten gemessen. Die Messungen erfolgen alle auf einem Referenzsystem. So k�nnen verschiedene Konfigurationen �ber ihre Performancewerte miteinander verglichen werden.
\section{Aufgabenstellung}\label{Aufgabenstellung}
In diesem Projekt soll die Abfrage-Performace f�r ein bestimmtes Szenario gemessen werden. Dazu ist ein Szenario auszuw�hlen, das im Bereich OLAP und Data Warehouse angesiedelt ist. F�r das Szenario sind geeignete Konfigurationen zu w�hlen. Konfigurationen unterscheiden sich in Art- und Anzahl von Indexen und in den Partitionierungsarten. Es wird angenommen, dass jede Konfiguration f�r eine bestimmte Art von Abfragen besonders geeignet ist. Es sollen verschiedenartige Abfragen gew�hlt werden, f�r die jeweils die Performance gemessen wird. Um die Performance messen zu k�nnen, ist ein ERD f�r ein gew�hltes Szenario anzufertigen. Die Tabellen des ERD sollen mit Testdaten gef�llt werden. Dazu ist ein Datengenerator anzufertigen. Die Abfragen werden auf die gef�llten Tabellen angewendet. Die Zeit, die ein Query braucht, um auf einer bestimmten Konfiguration ausgef�hrt zu werden, wird gemessen. Das Messergebnis wird mit der Annahme verglichen und ausgewertet.
\section{EasyMock}
Um den Datengenerator zu testen, wird EasyMock 3.0 verwendet. EasyMock ist ein Test-Framework zur dynamischen Generierung von Mock Objekten f�r Schnittstellen und Klassen. Mock-Objekte werden f�r Unit-Tests von Java-Programmen verwendet.

\subsection{Was ist ein Mock-Objekt?}
Beim Unit-Testing kollaborieren Units mit anderen Units. Die Kollaborateure werden durch Mock Objekte simuliert. Im Gegensatz zu einem Stub �berpr�ft das Mock-Objekt, ob es wie erwartet verwendet wurde. In einem Unit-Test werden Klassen bzw. Methoden isoliert von ihrer Umgebung getestet. Um die Testobjekte isoliert zu testen, m�ssen die Schnittstellen des Testobjekts durch Mock-Objekte ersetzt werden. Die Mock-Objekte sind Platzhalter f�r die echten Objekte. Das Verhalten eines dynamischen Mock-Objekts wird nicht in einer Klasse programmiert, sondern von dem Unit-Test aufgezeichnet. Es m�ssen keine Klassen von Hand geschrieben werden. Es muss auch kein Quellcode der Mock-Klassen, mit den echten Klassen synchron gehalten werden. Mock-Objekte werden bei EasyMock ''on the fly'' generiert, sind sicherer gegen Refactoring und damit besonders f�r Test Driven Development geeignet.

\subsection{Wie wird EasyMock benutzt?}
Um EasyMock zu benutzen, sind folgende Schritte n�tig:
\begin{enumerate}
	\item das Mock-Objekt von der Klasse / Schnittstelle die simuliert werden soll, erzeugen und dem zu testenden Objekt �bergeben,
	\item das erwartete Verhalten aufzeichnen,
	\item das Mock-Objekt auf Wiedergabe-Modus stellen,
	\item verifizieren, ob das Mock-Objekt auch so benutzt wurde, wie in Schritt zwei spezifiziert
\end{enumerate}
\cite{bib3}
\section{Dependency Injection}
Der Datengenerator verwendet das Architektur-Muster Dependency Injection um Abh�ngigkeiten zwischen Objekten zu minimieren. Dabei k�mmert sich das Objekt nicht mehr selbst um die Erzeugung seiner abh�ngigen Objekte. Diese Abh�ngigkeiten werden von einem Framework erstellt. Der Code des Objektes wird unabh�ngig von seiner Umgebung. Dadurch wird das Objekt leichter unit-testbar, da Abh�ngigkeiten zentral verf�gbar sind.

\subsection{Dependency Injection mit Google Guice}
In dem Datengenerator wird das Dependency Injection Framework Google Guice verwendet. Eine Abh�ngigkeit wird am besten �ber einen Konstruktor in ein Objekt injiziert. Daf�r braucht der Konstruktor nur mit @Inject annotiert werden.

Codebeispiel
\input{Ingo/DatengeneratorAnalyse}
\section{Entwurf}
\subsection{Komponentendiagramm}
Abbildung \ref{fig:Komponentendiagramm} auf Seite \pageref{fig:Komponentendiagramm} zeigt das Komponentendiagramm.

\begin{figure}[htp]
\centering
\includegraphics[width=1\textwidth]{Ingo/Bilder/Komponentendiagramm.png}
\caption{Komponentendiagramm}
\label{fig:Komponentendiagramm}
\end{figure}

\subsection{Entwurfsklassendiagramm der Generatorkomponente}
Abbildung \ref{fig:EntwurfGeneratorkomponente} auf Seite \pageref{fig:EntwurfGeneratorkomponente} zeigt das Entwurfsklassendiagramm der Generatorkomponente.

\begin{figure}[htp]
\centering
\includegraphics[width=1\textwidth]{Ingo/Bilder/EntwurfGeneratorkomponente.png}
\caption{Entwurf der Generatorkomponente}
\label{fig:EntwurfGeneratorkomponente}
\end{figure}

\subsection{Entwurfsklassendiagramm der DBWriterkomponente}
Abbildung \ref{fig:EntwurfDBWriterkomponente} auf Seite \pageref{fig:EntwurfDBWriterkomponente} zeigt das Entwurfsklassendiagramm der DBWriterkomponente.

\begin{figure}[htp]
\centering
\includegraphics[width=1\textwidth]{Ingo/Bilder/EntwurfDBWriterkomponente.png}
\caption{Entwurf der DBWriterkomponente}
\label{fig:EntwurfDBWriterkomponente}
\end{figure}

\subsection{Entwurfsklassendiagramm der Creatorkomponente samt Datamodel}
Abbildung \ref{fig:EntwurfCreatorUndDatamodelKomponente} auf Seite \pageref{fig:EntwurfCreatorUndDatamodelKomponente} zeigt das Entwurfsklassendiagramm der Creatorkomponente und dem dazugeh�rigen Datamodel.

\begin{figure}[htp]
\centering
\includegraphics[width=0.9\textwidth]{Ingo/Bilder/EntwurfCreatorUndDatamodelKomponente.png}
\caption{Entwurf der Creatorkomponente}
\label{fig:EntwurfCreatorUndDatamodelKomponente}
\end{figure}

\subsection{Architektur}
Der Generator hat eine zweischichtige Architektur (two tier architecture). Die Komponente Generator greift auf die Dienste der Komponente DBWriter zu. Auch sonst wurde in der Anwendung auf geringe Kopplung geachtet.

\subsection{Prepared Statement}
Die generierten Testdaten werden �ber sogenannte Prepared Statements in die Tabellen geschrieben. Prepared Statements enthalten Platzhalter f�r die eigentlich zu schreibenden Daten. Prepared Statements bieten sich u.a. dann an, wenn sich bei dem Statement nur die Parameterwerte unterscheiden. Prepared Statements bringen einen Geschwindigkeitsvorteil, da sie bereits im DBMS vor�bersetzt werden und der DBWriter mit sehr viel generierten Parameterwerten aufgerufen wird.

\subsection{Entwurfsmuster}
Der Creator ist als einfache Fabrik implementiert. Durch ihn werden konkrete Instanziierungen aus dem Clientcode entfernt und somit die Clients von konkreten Klassen entkoppelt (Dependency Inversion Principle).

In z.B. DBWriter konnte initDBWriter() als Template Method implementiert werden. Sie definiert die Schritte f�r die Initialisierung eines DBWriters, wobei die Unterklassen entscheiden, wie sie die abstrakte Hook-Methode prepareStatement() implementieren. Durch die Template Method werden die Highlevel Komponenten loadSQLDriver() und setUpDBConnection() von Lowlevel Komponente prepareStatement() entkoppelt. Die Lowlevel-Komponenten prepareStatement() kann sich in initDBWriter() reinh�ngen, und die Highlevel Komponenten initDBWriter() bestimmt, wann und wie prepareStatement() aufgerufen wird. Die Lowlevel Komponente prepareStatement() ruft die Highlevel Komponente dabei nie direkt auf (Hollywood Prinzip).

\newpage
\begin{flushleft}
\begin{thebibliography}{sotief}
\bibitem{bib1}{\textit{''Buchtitel''} Author}
\bibitem{bib2}{\it \begin{verbatim}http://www.google.de \end{verbatim}} 
\end{thebibliography}
\end{flushleft}


%\listoffigures

%\addcontentsline{toc}{section}{\bibname}
%\addcontentsline{toc}{section}{\listfigurename}
%\addcontentsline{toc}{section}{\indexname}\printindex


%\printglossaries
%\printglossary[type=symbols,style=long]
%\printglossary[type=main,title=Glossar,toctitle=Glossar]
\newpage
%\printglossary[type=\acronymtype]



\newpage
\begin{landscape}
\section*{Arbeitsaufteilung}
\begin{table}[h]

	\begin{center}
		\begin{tabular}{|l||c|c|c|c|c|c|}
	  	\hline
	  		 \textbf{Arbeit}		&	\textbf{Cristof Ochmann}	&	\textbf{Stefan L�ttke}	& \textbf{Ingo K�rner}  \\ \hline \hline
				  			Thema 1   	&     x              &                   &                     \\
				  			Thema 2	    &                   &          	x	      & 	             			\\ \hline
							  Thema 3   	&                   &                   &     x 			            \\ 
			  			 \hline
			  			 \hline
					Belegkapitel 		
					
%%%%%%%%%%%%%%%%%%%% Christof					
					&									     
					\ref{Einleitung}, \& \ref{Einleitung}
					
%%%%%%%%%%%%%%%%%%%% Stefan					
					&                       
					~\ref{Einleitung}
%%%%%%%%%%%%%%%%%%%% Ingo					
					&                     
					~\ref{Einleitung}, ~\ref{Einleitung}, ~\ref{Einleitung} - ~\ref{Einleitung}
					\\\hline\hline			       
		\end{tabular}			
	\end{center}
\end{table}
\end{landscape}


\newpage
\section*{Eigenst�ndigkeitserkl�rung} 

Hiermit erkl�re ich, dass ich diese Arbeit selbst�ndig verfasst habe.\\
Mir ist bekannt, dass jede Form des Plagiats mit der Note 5 (Betrugsversuch) bewertet wird. \\

\textbf{Ochmann, Christof} \\ 
\renewcommand{\baselinestretch}{1.25}\normalsize{
\noindent\hspace*{80mm}%
Unterschrift:\\}

\textbf{L�ttke, Stefan}\\
\renewcommand{\baselinestretch}{1.25}\normalsize{
\noindent\hspace*{80mm}%
Unterschrift:\\}

\textbf{K�rner, Ingo}\\
\renewcommand{\baselinestretch}{1.25}\normalsize{
\noindent\hspace*{80mm}%
Unterschrift:\\}

\end{document}
