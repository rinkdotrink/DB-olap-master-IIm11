\newpage
\chapter{Arbeitsaufteilung}

%WICHTIG! VOR DEM EDITIEREN BITTE LESEN:
%Unn�tige Tabelleneintr�ge (z.B. ebenfalls geschriebene Unterkapitel) entfernen um die LEsbarkeit zu verbessern!!!

%Tabelle 1:
\begin{table}[h] \begin{flushleft} \begin{tabular}{|l||c|c|c|c|c|c|}
\hline
\textbf{Arbeit}		&	\textbf{C. Ochmann}	&	\textbf{S. L�ttke}	& \textbf{I. K�rner}  \\ \hline \hline

Abstract   	      &                           &                         & 0       \\

\hline \hline
\end{tabular} \end{flushleft} \caption{Aufteilung vom Abstract} \end{table}

%Tabelle 2:

\begin{table}[h] \begin{flushleft} \begin{tabular}{|l||c|c|c|c|c|c|}
\hline
\textbf{Arbeit}		&	\textbf{C. Ochmann}	&	\textbf{S. L�ttke}	& \textbf{I. K�rner}  \\ \hline \hline

Das Projekt Wohnheimdatenbank &  & ~\ref{Wohnheim} & \\

\hline \hline
\end{tabular} \end{flushleft} \caption{Aufteilung von Kapitel 1} \end{table}

%Tabelle 3:

\begin{table}[h] \begin{flushleft}  \begin{tabular}{|l||c|c|c|c|c|c|}
\hline
\textbf{Arbeit}		&	\textbf{C. Ochmann}	&	\textbf{S. L�ttke}	& \textbf{I. K�rner}  \\ \hline \hline
Einleitung  &                   &          		      & ~\ref{Einleitung} \\
Aufgabenstellung&               &                   & ~\ref{Aufgabenstellung}     			            \\
Forschungsgegenstand&           &                   & ~\ref{RelevanzDesForschungsgegenstandes} \\ 
akt. Wissensstand&                   &                   & ~\ref{DerAktuelleWissensstand}  \\ 
Eingesetzte Datenbanken &  & \ref{kap2Datenbanken} & \\
Szenarien&  & & ~\ref{AnwendungsfaelleFuerDatenbankanwendungen}\\
Projektplanung &  & \ref{kap2Projektplanung} & \\
Datenbank &  &  & \\
Datenbankabfragen &  &  & \\
Anwendungsf�lle&                &                   & ~\ref{AnwendungsfaelleFuerDatenbankanwendungen}\\ 
EasyMock   	&                   &                   & ~\ref{EasyMock} 			            \\ 
Dependency Injection&           &                   & ~\ref{DependencyInjection}     \\ 
\hline \hline
\end{tabular} \end{flushleft} \caption{Aufteilung von Kapitel 2} \end{table}

\begin{table}[h] \begin{flushleft} \begin{tabular}{|l||c|c|c|c|c|c|}
\hline
\textbf{Arbeit}		&	\textbf{C. Ochmann}	&	\textbf{S. L�ttke}	& \textbf{I. K�rner}  \\ \hline \hline
Datengenerator&                 &                   & ~\ref{Datengenerator}         \\
\hline \hline
\end{tabular} \end{flushleft} \caption{Aufteilung von Kapitel 3} \end{table}

\begin{table}[h] \begin{flushleft} \begin{tabular}{|l||c|c|c|c|c|c|}
\hline
\textbf{Arbeit}		&	\textbf{C. Ochmann}	&	\textbf{S. L�ttke}	& \textbf{I. K�rner}  \\ \hline \hline
Die MySQL-Datenbank & ~\ref{MySQL-Datenbank} &  & \\
Optimierungsma�nahmen & ~\ref{Optimierungsmassnahmen} &  & \\
Indizes & ~\ref{Indizes} &  & \\
Partitionierung & ~\ref{Partitionierung} &  & \\
\hline \hline
\end{tabular} \end{flushleft} \caption{Aufteilung von Kapitel 4} \end{table}

\begin{table}\begin{flushleft}\begin{tabular}{|l||c|c|c|c|c|c|}
\hline
\textbf{Arbeit}		&	\textbf{C. Ochmann}	&	\textbf{S. L�ttke}	& \textbf{I. K�rner}  \\ \hline \hline
Testvorbereitung & ~\ref{Testvorbereitung} &  & \\
SELECT-Anweisungen & ~\ref{Select} &  & \\
Konfigurationen & ~\ref{Testvorbereitung} &  & \\
EXPLAIN-Anweisung & ~\ref{ExplainAnweisung} &  & \\
EXPLAIN PARTITIONS-Anweisung & ~\ref{ExplainPartitions}  &  & \\
Testdurchf�hrung & ~\ref{Testdurchfuehrung}  &  & \\
Annahmen und Vor�berlegungen & ~\ref{Testvorbereitung} &  & \\
Testergebnisse & ~\ref{Testergebnisse} &  & \\
Testauswertung & ~\ref{Testauswertung} &  & \\
Alle Messwerte & ~\ref{Alle} & ~\ref{Alle} & ~\ref{Alle}\\
Zusammenfassung &  & \ref{Summary} & \\
\hline \hline
\end{tabular} \end{flushleft} \caption{Aufteilung von Kapitel 5} \end{table}

\begin{table}\begin{flushleft}\begin{tabular}{|l||c|c|c|c|c|c|}
\hline
\textbf{Arbeit}		&	\textbf{C. Ochmann}	&	\textbf{S. L�ttke}	& \textbf{I. K�rner}  \\ \hline \hline
Ausblick &  & \ref{Ausblick} & \\
\hline \hline
\end{tabular} \end{flushleft} \caption{Aufteilung von Kapitel 6} \end{table}